\section{Injector}

Injector setup started with magnetic elements set according to 
the design values. First, 3~GHz beam was commissioned, from which 1.5~GHz was derived.
The aim of the injector setup was a good capture efficiency, shortest bunches with energy spready below 1\%.
All three parameters could not be maximized simoultaneusly, therefore a compromise had to be found.
First, maximum caputure and transmission was targeted. 
In the second step a setting with a minimum energy spread was found. 
Finally, phases were shifted towards setting giving shorter bunches.
The ultimate goal was achieving minimum energy spread at the end of the linac, which depended on both, 
initial bunch length and energy spread.
Therefore, the final injector tunining was performed with the measurements at CTS.
It was also important to have small sensitivity of the produced beam current
on the RF phases and amplitudes. Otherwise jitter and drift of the RF sources 
were transmitted to the beam energy via the beam current, which is a consequence of 
the fully loaded linac operation. 


Naturally, for the machine safety the initial setup was carried with
the shortest pulse visible for the diagnostic devices, which was around 250~ns.
%For the machine safety shortest pulse visible for the diagnostic devices was used, that was around 250~ns.
In all accelerating structures, but not in the pre-buncher and buncher,
RF phase had a parabolic shape due to the pulse compression.
For the initial steps the part around the maximum of the parabola was selected to avoid 
phase and energy variation along the beam pulse.

The bend \texttt{CL.BHA0425-S}, redirecting the beam to the the first dump in spectrometer line on girder 4, 
was set according to the expected beam energy at this location (between -120~A and -140~A for the 4~A drive beam).
The setup procedure for the 3~GHz bunch spacing comprised the following steps:
\begin{enumerate}
\item 
The beam was threaded until the pre-bunching cavity \texttt{CL.PBU0245} using one-to-one steering. 
Its tranverse profile was verified with \texttt{CL.MTV0165} screen and if needed eventually solenoids and steerers were adjusted. 
\item
Attenuator of the prebuncher was set to minimum becuase 
for 3~GHz bunch spacing the simulations predicted, and the experience confirmed it, 
that higher the power shorter the bunch is.
Phase of pre-buncher was tuned to obtain maximum %(knob \texttt{CK.PBU0245-PHAS})
signal in \texttt{CL.BPR0290W}, which corresponded to the shortest bunch length.
This device was also sensitive to beam current and position, so it had to be watched that they do not change during the optimizations.
\item 
The orbit was corrected until the first accelerating cavity \texttt{CL.ACS0305}.
\item 
Its phase was tuned using the MKS03 phase loop knob \texttt{CK.MKS03-PLOOP} to provide maximum loading visible on signal \texttt{CK.PEI0305A}.
\item 
The setting of the spectrometer bend \texttt{CL.BHA0425-S} was adjusted to find the beam in the center of the screen \texttt{CLS.MTV0440}
\item
Solenoids (girders 1 to 3) and quadrupoles (girder 4) were adjusted to provide maximum transmission 
as recorded by the BPE's and BPM's.
\item
Quardupoles were set to obtain the most narrow in the horizontal plane spot on CLS.MTV0440, 
but in the vertical plane still well confined within the screen aperture. 
It was done in order to minimize the contribution of the $\beta$-function to 
measured total beam size in the spectrometer that was needed for energy spread optimizations.
\item
Energy spread was measured using the segmented dump device CLS.SDU0442 in the spectrometer. 
Scans in function of the MKS03 klystron phase were done to find a minimum.
\item
The beam was sent towards the linac: quadrupoles and the bends of the compressor chicane 
(\texttt{CL.BHA0425-S} and \texttt{CL.BHA0430-S}) were set to the design values.
Steerers and eventually quadrupoles were adjusted to maximize the transmission 
up to position monitor CL.BPM0502. 
In case of difficulties beam profile measurement with screen CLS.MTV0500 
was also used in the manual optimization.
\item
Phase of accelerating structures at girder 3 and amplitude of the compressing chicane was adjusted to minimize 
the bunch length (maximize BPR waveguid signal \texttt{CL.BPR0475W}) while keeping energy spread below 1\% r.m.s.
as measured with segmented dump \texttt{CLS.SDU0442}. 
Usually it was around 7 degrees from the full beam loading condition.
\item 
The pulse length was extended to its maximum, i.e. 1.3~\textmu s. 
Phase program of klystron MKS02 was set identical to measured phase sag of MKS03 compressed pulse. 
Flattness of the beam energy and of the bunch length along the pulse was verified with 
the segmenented dump and \texttt{CL.BPR0475W}, respectively.
Because the callibration of the phase measurements and phase shifters was not perferct
it needed further adjustment. The most efficient was autumatic precudure
that corrected the MKS02 phase to find \texttt{CL.BPR0475W} constant.
\item
Quadrupole scan with CLS.MTV0500 measured Twiss parameters at CL.QFA0460.
Steering through the solenoids and their settings were adjusted to minimize the emittance. 
\item
Using MADX model the measured Twiss parameters were propagated to the exit of the last solenoid.
MADX was also used to rematch the quadrupoles settings to correct the beta-beating.
Usually 2 or 3 iterations were needed to find satisfactory result. 
\end{enumerate}

1.5~GHz bunch spacing was enabled powering the sub-harmonic bunchers.
Initially phase switches were disabled. 
The goal was to have minimum number of differences between the 3 and 1.5~GHz beams
to easily switch between the two and to avoid doubling the time for the setup.
This way also any improvement found with one type of the beam 
could be immediately copied to the other one.

\begin{enumerate}
\item
SHB1 was powered with the nominal power. 
Its phase was adjusted to maximize signal on \texttt{CL.BPR0290W} to 
assure that maximum beam current is kept captured and minimum number of electrons
is back scattered what eventually was disturibing readout of many signals.
\item
SHB2 was powered with the nominal power. 
The design predicted that the optimum phase should correspond to more or less no beam loading condition.
The phase was tuned to find maximum signal at the output of SHB3
because when this structure was not powered the maximum amplitude of the produced RF 
corresponded to the shortest bunches in this cavity. 
\item
Input of the 3~GHz pre-buncher was attenuated so it could be used as a bunch length monitor.
\item
SHB3 was powered and its phase tuned to find maximum beam loading from the 3~GHz pre-buncher.
\item
After powering back the pre-buncher, phases of all SHBs were shifted together to find maximum
of \texttt{CL.BPR0290W} signal.
%Of course there were two setting fullfilling this condition. 
%The one producing higher signal in \texttt{CL.BPR0290W} was selected.
\item
Amplitude and phase of prebuncher, as well as of the three SHBs and klystron MKS02 (powering bre-buncher and buncher),
were scanned to find best caputure and shortest bunches at the end of the injector in \texttt{CL.BPR0475W}
while keeping the energy spread reasonably small.
It was an iterative procedure involving repeated energy spready and bunch length measurements
while varying each parameter in vicinity of its set point.
\item
The gun emmission was tuned to find the same beam current in the linac as in the 3~GHz reference.
If the change was substantial (more then 5\%) the setup procedure had to be repeated to assure the optimum phase settings. 
\item
Twiss parameters were verified with quadrupole scans at girder 5 and 10 
and eventually beta beating was corrected.
\item
Phase switches in SHBs were enabled and its setup procuedure is detailed further down. 
A priori it should not change the beam dynamics,
but in reality the injector had to be further optimized to reduce transient effects occuring after each switch.
\item
The final fine tuning of the injector was done minimizing the energy spread measured 
at the end of the linac in CTS segmeneted dump.
\end{enumerate}


\todo[inline]{Talk to davide http://elogbook.cern.ch/eLogbook/eLogbook.jsp?date=20160411\&lgbk=100}
The procedure 
%http://elogbook.cern.ch/eLogbook/eLogbook.jsp?shiftId=1073715
% the same as above
%http://elogbook.cern.ch/eLogbook/eLogbook.jsp?date=20160411&lgbk=100

1) first SHB with switches. not touching phase. 

remove switches on second SHB. 

remove RF from third SHB. 

-> verify zero crossing on loading of second SHB. 

-> measure signal on third SHB. 


2) start switches on second SHB. 

-> find the 180-side to set second SHB2 to maximise loading on third TWT 

2b) remove swithces on second SHB and re-verify zero crossing to the new phasee if you needed to change it. 

3) without the beam, with switches back on second SHB, vefify the synchronization between switches induced by the beam and the one of the RF. 
The same with beam on. 

4) insert third SHB without switches. 

-> find zero crossing. 

4b) attenuate the prebuncher and log the power produced by the beam in this situation. 

4c) switch off the beam and start switches on third TWT. 

-> verify timing of switches in third SHB. 

5) with switches and RF in all SHB start the beam 

-> verify the 180-degree phase of third TWT by looking at output of prebuncher. 

6) set back prebuncher. 






