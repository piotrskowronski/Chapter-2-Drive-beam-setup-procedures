\section{TL2}

Setup of the \ac{TL2} transfer line naturally followed the same 
type of procedure as descibed for \ac{TL1}.
Here particularly difficult was transport through the chicanes
where bends were connected in series and
there were 13 BPMs and 12 orbit correctors for almost 40 magnets.
Clearly the principal reason was again poor quality of the quadrupoles
iherited from CELCIUS machine.
When the settings of the bending magnets was established on a non focused beam
and afterwards the quadrupoles were switched on, the beam orbit changed dramatically.
It was not possible to arrive to a setup where the beam was well transported
and the orbit was not sensitive to the quadrupole powering.
This of course also induced dispersion errors.
Needless to mention that alignment was checked and confirmed to be within 100~\textmu m.

\textbeta-beating was measured with quadrupole scans at the beginning and at the end of the line
and as usuall corrections were computed with MADX. 
Contrary to the earlier parts of the machine this procedure was not very effective.
In this case, corrections performed at the first sections of the TL2 line 
(where dispersion was zero) did not converge well to the required values. 
Eventually the correction was implemented at the end of the TL2 line
with the two last doublets.
Matching to the TBL was not problematic, contrary to TBTS/TBM with 
simular issues as in TL2.







