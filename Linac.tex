\section{Linac and CT line}

RF pulse comression setup?


\begin{enumerate}

\item
Bend CL.BHB1040 was set to direct the beam towards CLS dump
assuming the nominal beam energy at this point of the linac.
\item
Iterating for each girder, phases of the cavities were adjusted to find maximum beam loading
(minimum power measured at structures output)
and orbit corrected manually to get maximum transmission.
In rare cases when losses were still observed 
quadrupoles at girders 5 and 6 were manually adjusted to minimize the losses.
Alternatively all the quadrupoles were scaled up or down.
\item
Powering of CL.BHB1040 bend was adjusted to find the beam in the middle 
of screen \texttt{CLS.MTV1050} and of segmented dump \texttt{CLS.SDU0442}.
Beam energy was calculated from this value using the magnet excitation curve.
It was important to verify that the orbit before the bend is well centered
to minimize the error on the energy measurement. This was achieved not only by 
looking at the BPM readings but also at orbit sensitivity to changes
of powering of the three upstream quadrupoles.
\item
Energy gain of each accelerating structure was estimated from their measured input power
and the beam current using a dedicated cumputer program.
Having beam energy profile along the linac powering of the quadrupoles was adjusted.
\item
Quadrupole scan with \texttt{CLS.MTV1026} revealed Twiss parameters at quadrupole CL.QDB1005.
\item
MADX was used to propagate \textbeta-function towards the injector and
to rematch quadrupoles from girder 6 to 9 to correct the \textbeta-beating.
\item
Automatic \ac{DFS} was performed.
\item
Phasing of the accelerating cavities was adjusted to minimize energy spread
measured with the segmented dump. 
To measure the energy spread the beam spot observed at the screen in the spectrometer was 
first minimized in the horizontal plane using the three quadrupoles on girder 10 
assuring that it is not too big in the verical one.
The energy spread measurement didn't have enough resolution
to be sensitive to a single cavity adjustments within the required resolution.
Instead, phases of all the accelerating structures in the linac were shifted together 
from the values corresponding to the maximum beam loading.
Usually it was around 3-4 degrees.
\item
At this point the injector was fine tuned to eventually further reduce emittance and energy spread
because the measurements at this location had better resolution compared to the ones at the end of the injector.
In case any improvements were implemented the procedure for the linac setup was repeated.
\item
Using the results of quadrupole scan the powering of the quadrupoles on girders 10 to 15 
was rematched using a dedicated MADX script. 
\item
The beam was directed towards the CTS dump. The same way as in the first part of the linac,
phases of the accelerating structures and orbit correctors were setup girder after girder.
\item
Twiss parameters were measured with quadrupolar scan using screen \texttt{CT.MTV0435}
and \textbeta-beating corrected adjusting the upstream CT line quadrupoles.
\item
Twiss parameters measured at CTS with screen CT.MTV0550 were checked
and \textbeta-beating was corrected using the quadrupoles located downstream of CT.MTV0435.
It is important to mention here that often the results of quadrupole scans in screens
CT.MTV0435 and CT.MTV0550 did not match. 
Of course the optics was verified with response matrix and 
the profiles monitors were also showing no issue with callibration.
It was CT.MTV0550 that was eventually more trusted because
measurements with CT.MTV0435 were often changing for different ranges of the scans.
The most plausible explanation of the issue is related to strongly diverging beam at this location
(\textalpha-parameter bigger then 5) what made finding good range for the scan very difficult.
For a good scan a waist needs to be created in the probed plane and its minimum 
swept in steps from one side of the screen to the other.
Maximum range of phase advances needs to be probed to contain well the phase space ellipse.
However, if the beam has big \textalpha-parameter the produced waist is very steep.
It has two adverse consequences. 
First, the minimum beam size is very small reaching the limit of granularity of the CCD camera
and the realtive error of beam size measurement becomes large.
Second, at some locations the beam size quickly becomes too large to fit within the aperture.
This eventually limits range of scanned phase advances.
The situation becomes even more difficult if in the opposite plane absolute of \textalpha-function
is also large because the beam is likely to become too big also in this plane.
\item
The energy spread was measured and phases of the accelerating structures 
in the second part of the linac were tuned to minimize it.
\end{enumerate}
